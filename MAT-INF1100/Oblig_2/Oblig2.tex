\documentclass[,norsk]{article}
\usepackage[utf8]{inputenc}
\usepackage[T1]{fontenc}
\usepackage{babel}
\usepackage{amsmath}
\usepackage{amsfonts}
\usepackage{amsthm}
\usepackage[colorlinks]{hyperref}
\usepackage{listings}
\usepackage{graphicx}
\usepackage{varioref}
\lstset{
	tabsize=4,
	rulecolor=,
	language=python,
        basicstyle=\scriptsize,
        upquote=true,
        aboveskip={1.5\baselineskip},
        columns=fixed,
	numbers=left,
        showstringspaces=false,
        extendedchars=true,
        breaklines=true,
        prebreak = \raisebox{0ex}[0ex][0ex]{\ensuremath{\hookleftarrow}},
        frame=single,
        showtabs=false,
        showspaces=false,
        showstringspaces=false,
        identifierstyle=\ttfamily,
        keywordstyle=\color[rgb]{0,0,1},
        commentstyle=\color[rgb]{0.133,0.545,0.133},
        stringstyle=\color[rgb]{0.627,0.126,0.941}
        }

\title{MAT-INF 1100 Obligatorisk oppgave 2}
\author{Kenneth Ramos Eikrehagen}
\begin{document}
\maketitle
\tableofcontents

\section{Oppgave 1.}
\\
\textbf{a)}\\
Her brukte jeg at akselerasjon er den deriverte av farten. Det betyr at akselerasjonen er $\frac{fartsendring}{tids endring}$ altså $a = \frac{\delta x}{\delta t}$ Jeg lagde to arrayer av fart- og tidsverdiene slik at jeg kunne indeksere inn endring i fart over endring av tid og lagret dette i en array for akselerasjon som jeg senere skulle plotte. \\
\textbf{b)}\\
Her brukte jeg numerisk integrasjon, nærmere bestemt trapes metoden. Fordi jeg vet at integralet av fart = posisjon, og trapes metoden er en god tilnærming til den virkelige grafen. \\
\textbf{c)}\\
Her skulle vi plotte a) og b), jeg laget derfor en klasse der jeg kunne lett hente inn de funksjonene jeg ville bruke og det ble enklere for meg å plotte resultatet. Jeg syntes også koden ser penere ut på denne måten. Jeg plottet også begge grafene i et vindu. \\
Jeg har lagt ved python koden min:
\lstinputlisting{algoritme_a.py}
\begin{figure}
\includegraphics[width = 1\textwidth]{aks_pos.png}
\caption{Jeg har plottet begge grafene i samme plott \label{aks}}
\end{figure}
Den øverste grafen i figur\vref{aks} viser hvordan akselerasjonen er som funksjon av tiden t. Den nederste grafen i figur\vref{aks} viser hvordan posisjonen endrer seg med tiden.
\\
\section{Oppgave 2.}
\textbf{a)}\\
\begin{equation}
x' + x^2 =1,\text{ } x(0)=0
\end{equation}\\
Dette kan også skrives som $$\frac{d}{dt}x(t) = 1 - x(t)^2$$ 
Deler vi med $1-x(t)^2$ og tar integralet på begge sider får vi en ligning som ser slik ut: $$ \int{\frac{\frac{d}{dt}x(t)}{1-x(t)^2} dt= \int{1 dt}$$ \\
Jeg ser at venste siden blir $t + c_1$, jeg løser så høyre siden og slår sammen:
\begin{equation}
\int{\frac{\frac{d}{dt}x(t)}{1-x(t)^2} dt= \int{\frac{dx(t)}{1-x(t)^2} = artanh(x(t)) + c_2
\end{equation}\\
$$artanh(x(t)) + c_2 = t + c_1$$
$$artanh(x(t)) = t + c_1 - c_2$$
$$x(t) = tanh(t + c)$$
Putter inn $x(0) = 0$ får at $tanh(c)=0 \Rightarrow c=0$, $x(t) = tanh(t)$ Vet at $tanh(t) = \frac{e^t-e^{-t}}{e^t+e^{-t}}$ kan nå skrive $x(t)$ på en annen måte. $$x(t) = \frac{e^t-e^{-t}}{e^t+e^{-t}}$$
\\Her har jeg brukt at $\int{\frac{1}{1-x^2} dx = artanh(x) + c$ som det står om i kap 7.7 side 387 i kalkulus.\\
\textbf{b)}\\
Jeg har brukt følgende algoritme til å løse denne oppgaven.\\
\begin{equation} n=5 ,\text{ }  h = \frac{2}{5},\text{ } k = \left\{0,1,2,3,4,5\right\}
\end{equation}
$$x_{k+1} = x_k + h\cdot f(t_k, x_k)$$ $$t_{k+1} = a + (k+1)\cdot h$$
Legger ved python koden:
\lstinputlisting{Euler_vs_exact.py}
\begin{figure}
\includegraphics[width=1\textwidth]{Euler_vs_exact.png}
\caption{I dette plottet ser vi Euler metoden mot den eksakte løsningen\label{Euler}}
\end{figure}
På figuren\vref{Euler} ser vi hvorden Euler metoden bruker tangenten i $x_i$ for å finne en tilnærming til den eksakte grafen.\\
\textbf{c)}\\
Jeg modifiserte litt på algoritmen jeg brukte i oppgave b). \\
\begin{equation} n=5 ,\text{ }  h = \frac{2}{5},\text{ } k = \left\{0,1,2,3,4,5\right\}
\end{equation}
$$x_{k+\frac{1}{2}} = x_k + \frac{h}{2}\cdot f(t_k, x_k)$$
$$x_{k+1} = x_k + h\cdot f(t_K, x_{k+\frac{1}{2}})$$
$$t_{k+1} = a + (k+1)\cdot h$$
Legger ved python koden:
\lstinputlisting{diff3.py}
\begin{figure}
\includegraphics[width=1\textwidth]{Euler_m_e.png}
\caption{I dette plottet ser vi Euler metoden og Eulers midtpunkt metode mot den eksakte løsningen\label{Emidt}}
\end{figure}
\textbf{d)}\\
Jeg ser ut ifra grafene jeg har fått når jeg har plottet at for $0\leq x(t^*) < 1$ er voksende uansett hvilken verdi $t^*$ har, innenfor de begrensiningene som er satt. Hvis vi bruker grafen til tanh(t), som vi ser på figur \vref{tanh}, at den går mot 1 for $t^* \in [1, \infty)$
\begin{figure}
\includegraphics[width=1\textwidth]{tanh.png}
\caption{Grafen til den hyperbolske funksjonen tanh(x)\label{tanh}}
\end{figure}
\end{document}

