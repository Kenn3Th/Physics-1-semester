\documentclass[,norsk]{article}
\usepackage[utf8]{inputenc}
\usepackage[T1]{fontenc}
\usepackage{babel}
\usepackage{amsmath}
\usepackage{amsfonts}
\usepackage{amsthm}
\usepackage[colorlinks]{hyperref}
\usepackage{listings}
\lstset{
	tabsize=4,
	rulecolor=,
	language=python,
        basicstyle=\scriptsize,
        upquote=true,
        aboveskip={1.5\baselineskip},
        columns=fixed,
	numbers=left,
        showstringspaces=false,
        extendedchars=true,
        breaklines=true,
        prebreak = \raisebox{0ex}[0ex][0ex]{\ensuremath{\hookleftarrow}},
        frame=single,
        showtabs=false,
        showspaces=false,
        showstringspaces=false,
        identifierstyle=\ttfamily,
        keywordstyle=\color[rgb]{0,0,1},
        commentstyle=\color[rgb]{0.133,0.545,0.133},
        stringstyle=\color[rgb]{0.627,0.126,0.941}
        }

\title{MAT-INF 1100 Obligatorisk oppgave 1}
\author{Kenneth Ramos Eikrehagen}
\begin{document}
\maketitle
\tableofcontents
\section{Oppgave 1.} 
\textbf{a)}\\
Jeg lagde et dataprogram til differensligningen: 

$x_{n+2} - 2 x_{n+1} - x_{n} = 0$ med  $x_0 = 1$ og $x_1 = 1$.

Jeg brukte at  $x_{n+2} = 2x_{n+1} + x_n$ i programmet.
\lstinputlisting{oblig1.py}
\\
\textbf{b)}\\
Her er hvordan programmet ser ut når startverdiene endres til $x_0 =
1$ og $x_1 = 1 - \sqrt2$
\lstinputlisting{oblig1_2.py}
\textbf{c)} \\ 
Jeg bruker at $x_n=C\times r^n$ på ligningen $x_{n+2} - 2 x_{n+1} - x_{n} =0$ som gir den karakteristiskeligningen $r^2-2r-1$. Jeg løser den ved hjelp av abc formelen:
\begin{equation}
\frac{2_-^+\sqrt{2^2-(4\times1\times(-1))}}{2} =\frac{2_-^+\sqrt8}{2} = \frac{2_-^+2\sqrt2}{2} =
1_-^+\sqrt2
\end{equation}
Jeg får to røtter $r_1=1-\sqrt2$ og $r_2=1+\sqrt2$ \\
Ut i fra Lemma 4.1.4 blir den generelle løsningen til $x_{n+2} - 2 x_{n+1} - x_{n} =0$ $$x_n = C(1-\sqrt 2)^n + D(1+\sqrt2)^n$$ 
Får å få en endelig løsning setter jeg inn startverdiene $x_0=1$ og $x_1=1-\sqrt2$ inn i den
generelle løsningen. $$1 = x_0 = C(1-\sqrt2)^0 + D(1+\sqrt2)^0 = C +
D\Rightarrow C = 1 -D$$
$$1-\sqrt2 = x_1 = C(1-\sqrt2)^1 + D(1+\sqrt2)^1 = (1-D)(1-\sqrt2) +
D(1+\sqrt2)$$ $$= (1 - \sqrt2 - D +D\sqrt2) + D + D\sqrt2 = 1 -\sqrt2 +
2D\sqrt2 = 2D\sqrt2 \Rightarrow D = 0 \Rightarrow C =1 - 0$$
Setter inn C og D i den generelle løsningen og får :
$$x_n = 1\times(1-\sqrt 2)^n + 0\times(1+\sqrt2)^n = (1 - \sqrt2)^n$$
\textbf{d)}\\
Løsningen i c) stemmer ikke med beregningene mine i b). Dette er pga
avrundnings feil.  I c) ser man at $(1+\sqrt2)^n$ forsvinner, altså
blir akkurat 0.0, men på datamaskinen skjer det en avrundnings feil
her. I steden for å bli 0.0 blir svaret $10^{-16}$ dermed faller
ikke den bort. Dette medfører at det blir feil når n blir stor.
Avviket i starten er liten, men alikvel vokser den. Vi ser
allerede når n = 20 ($x_{20}$) at svaret endre seg fra å gå mot null
til å stige. Når jeg kommer til n = 40 ($x_{40}$) er avviket allerede på
$6.99\cdot 10^{-2}$ på n = 60 ($x_{60}$) er avviket på $3.16\cdot 10^6$! på
n = 100($x_{100}$) er avviket kommet på $6.48\cdot 10^{21}$! Avrundningsfeilen
har store konsekvenser for denne følgen. Jeg legger ved en kode jeg brukte for å studere feilen nærmere.
\lstinputlisting{oblig1_1.py}
\section{Oppgave 2.}
\textbf{a)}
\lstinputlisting{oblig1_3.py}
Jeg programmer i python og da trenger jeg ikke bruke flyttall, dette
er på grunn av at python setter opp presisjonen når tallene blir
store. Her er ikke ''integer overflow'' så farlig, det som skjer er at
kjøringen av programmet går saktere enn vanlig. Heltall aritmetikk er
mer presis og svarerne man får er eksakte, og det er lett å se om noe
går feil (resultatene er fullstendig feil!). Her er det greit å bruke
heltall siden binomialkoeffisientene er heltall og divisjonen gir
ingen rest. Hvis jeg hadde brukt et annet programerings språk kan det tenkes
at jeg burde ha brukt flyttall og tilnærminger for å forhindre ''overflow''. \\
Svarene jeg fikk ser man under koden. \\
\textbf{b)}\\
Ja. Dette kan skje når enten i eller n blir for stor i forhold til
hverandre. I mitt program i a) så blir det overflow når n blir mye større en i
selv om binomialkoeffisienten er innen for det største flyttallet som
kan representere på min maskin. Jeg får memory error når jeg prøver
f.eks $n = 2^{63}$ og $i = 2$ eller $n = 2*10^9$ og i = 2 \\
\textbf{c)}\\
\begin{equation}\label{eq:binomdef}
\binom{n}{i}=\frac{n!}{i!\,(n-i)!}
=\frac{n(n-1)(n-2)\cdot...\cdot(n-i+1)(n-i)(n-i-1)(n-i-2)\cdot3\cdot2\cdot1}{i!
  (n-i)(n-i-1)(n-i-2)\cdot...\cdot3\cdot2\cdot1}
\end{equation}
Jeg forkorter og får:
\begin{equation}\label{eq:binomdef}
\frac{n(n-1)(n-2)\cdot...\cdot(n-i+1)}{i!}
\end{equation}
Ved hjelp av produktnotasjon kan vi derfor skrive $\binom{n}{i}$ som
\begin{equation}\label{eq:altdef}
\binom{n}{i}=\prod_{j=1}^{i} \frac{n-i-j}{j} = \frac{n(n-1)(n-2)\cdot...\cdot(n-i+1)}{i!}
\end{equation}
Denne metoden er bedre å bruke når n er mye større enn i. F.eks $n = 2^{63}$ og $i = 2$
. Den forrige er best å bruke når i er mye større enn n.
\section{Oppgave 3.}
\textbf{a)}\\
\lstinputlisting{oblig1_opg3.py}
Programmet importer funksjonen random.\\
Den oppgir variabler (antfeil, x0, y0, z0, feil1, feil2) som er tomme, altså med verdi 0.0, og N til 10 000.\\
x = random() genererer et tilfeldig flyttall i intervallet [0.0, 1.0)\\
y = random() genererer et tilfeldig flyttall i intervallet [0.0, 1.0). \\
res1 adderer og subtraherer de tilfeldige flyttallene x og y, før den
multipliserer resultatene.$(x + y)*(x - y)$\\
res2 kvadrerer de tilfeldige flyttallene x og y også subtraherer
de.$x^2 - y^2$\\
Hvis res1 er forskjellig fra res2 legger programmet til 1 i
antfeil. Tallet x blir satt til $x_0$ og tallet y blir satt
til $y_0$. res1 blir satt til feil1 og res2 blir satt til feil2.
Programmet skriver ut antall feil som har forekommet i prosent,
skriver også ut x og y verdien på den siste feilen som ble gjort samt differansen mellom
res1 og res2 på den siste feilen.\\
\textbf{b)}\\ 
En mulig forklaring kan være at det gjøres flere opperasjoner i det
første programmet i forhold til det andre.
Siden i første programmet skal programmet addere x og y, så subtrahere x
og y før den multipliserer svarerene fra addisjonen og
subtraksjonen.$(x + y)\times(x -y)$ og den skal sammenligne dette med
$x^2 - y^2$ Altså 6 opperasjoner som gir 6 muligheter for feil. \\
I det andre programmet skal den dele x på y.$\frac{x}{y}$ Og sammenligner dette
med 1 over y delt på x. $\frac{1.0}{\frac{y}{x}}$ Her er det altså 3
opperasjoner som medfører 3 muligheter for feil.  Jeg tror dette kan
være en grunn til at feilen blir mindre i program 2.\\
En annen forklaring er at når man adderer og subtraherer på en
datamaskin kan det fort hende at noen tall forsvinner, spesielt hvis
disse tallene er veldig like, denne effekten kalles
''cancellation''. Cancellation forekommer ikke når man skal dividere
eller multiplisere, den verste feilen som kan skje når man gjør en av disse
operasjonene er at det siste sifferet er feil. Siden både addisjon og
subtraksjon forekommer i program 1 og kun divisjon forekommer i
program 2 er nok dette mest sansynlig grunnen til at det er mindre
feil i program 2.

\end{document}
